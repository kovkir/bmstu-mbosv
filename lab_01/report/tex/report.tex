\documentclass{bmstu}

\begin{document}

\makereporttitle
{Информатика и системы управления (ИУ)}
{Программное обеспечение ЭВМ и информационные технологии (ИУ7)}
{\textbf{1}}
{Математические основы верификации ПО}
{Знакомство с языком Promela}
{}
{ИУ7-42М}
{К.Э. Ковалец}
{О.В. Кузнецова}


\setcounter{page}{2}
% \renewcommand{\contentsname}{Содержание} 
% \tableofcontents


\chapter{Выполнение лабораторной работы}

\textbf{Promela (Process Meta Language)} --- это язык программирования, разработанный для описания и верификации распределённых и параллельных систем. Он используется в связке с инструментом SPIN (Simple Promela Interpreter) для моделирования, проверки корректности и поиска ошибок в системах с параллельными процессами. SPIN позволяет выполнять формальную верификацию моделей, написанных на языке Promela, и проверять их на наличие различных типов ошибок, таких как взаимоблокировки, гонки данных и другие проблемы, связанные с параллелизмом.

\section{Задание}

Для небольшого фрагмента программы (10-15 строк кода, в которых есть изменение значений
переменных) необходимо описать модель этой программы на Promela и изучить её (SPIN).

Отчёт должен содержать:
\begin{itemize}
  \item фрагмент кода;
  \item описание модели;
  \item перечисление множества состояний и текстовое пояснение причин переходов между состояниями;
  \item граф переходов между состояниями модели;
  \item вывод по работе.
\end{itemize}

\clearpage
\section{Фрагмент кода}

Код программы на языке Promela приведен в листинге \ref{lst:main}.

\mylisting[c]{main.pml}{}{Пример программы на языке Promela}{main}{}

\section{Описание модели}

Данная модель описывает процесс, который определяет тип треугольника на основе длин его сторон.

Модель включает следующую функциональность:
\begin{itemize}
  \item Процесс \texttt{check\_triangle\_type} принимает три целочисленных параметра (\texttt{a}, \texttt{b} и \texttt{c}), представляющих стороны треугольника.
  \item Сначала проверяется, могут ли данные стороны образовать допустимый треугольник, используя теорему о неравенстве треугольника.
  \item Если стороны образуют допустимый треугольник, они сортируются так, чтобы \texttt{c} всегда была самой длинной стороной.
  \item Затем проверяется тип треугольника:
  \begin{itemize}
     \item Прямоугольный треугольник: если \(a^2 + b^2 == c^2\)
     \item Остроугольный треугольник: если \(a^2 + b^2 > c^2\)
     \item Тупоугольный треугольник: в остальных случаях
  \end{itemize}
  \item Если стороны не образуют допустимый треугольник, выводится сообщение об этом.
  \item Процесс \texttt{init} демонстрирует использование \texttt{check\_triangle\_type} с различными наборами длин сторон:
  \begin{itemize}
    \item (3, 4, 5): Прямоугольный треугольник
    \item (5, 5, 8): Тупоугольный треугольник
    \item (3, 3, 4): Остроугольный треугольник
    \item (1, 2, 3): Не треугольник
  \end{itemize}
\end{itemize}

\clearpage
\section{Перечисление множества состояний}

Множество состояний для процесса \texttt{check\_triangle\_type} включает следующие состояния:

\begin{itemize}
  \item \textbf{S0}: Начальное состояние, где начинается процесс \texttt{check\_triangle\_type}.
  \item \textbf{S1}: Проверка условия существования треугольника (\texttt{a + b > c \&\& a + c > b \&\& b + c > a}).
  \item \textbf{S2}: Стороны не образуют треугольник.
  \item \textbf{S3}: Стороны образуют треугольник, сортировка сторон.
  \begin{itemize}
    \item \textbf{S31}: Проверка, является ли сторона \texttt{a} самой длинной. Если да, меняем \texttt{a} и \texttt{c}.
    \item \textbf{S32}: Проверка, является ли сторона \texttt{b} самой длинной. Если да, меняем \texttt{b} и \texttt{c}.
    \item \textbf{S33}: Сторона \texttt{c} уже является самой длинной, сортировка не требуется.
  \end{itemize}
  \item \textbf{S4*}: Проверка типа треугольника (\texttt{a² + b²} по отношению к \texttt{c²}).
  \item \textbf{S5*}: Прямоугольный треугольник (\texttt{a² + b² == c²}).
  \item \textbf{S6*}: Остроугольный треугольник (\texttt{a² + b² > c²}).
  \item \textbf{S7*}: Тупоугольный треугольник (\texttt{a² + b² < c²}).
  \item \textbf{S8**}: Завершение процесса.
\end{itemize}

Переходы между состояниями происходят по следующим причинам:

\begin{itemize}
  \item \textbf{S0 $\rightarrow$ S1}: Начало процесса проверки треуголника по трем сторонам.
  \item \textbf{S1 $\rightarrow$ S2}: Условие существования не выполнено, стороны не образуют треугольник.
  \item \textbf{S1 $\rightarrow$ S3}: Условие существования выполнено, стороны образуют треугольник.
  \item \textbf{S3 $\rightarrow$ S31}: Проверка, является ли сторона \texttt{a} самой длинной.
  \item \textbf{S3 $\rightarrow$ S32}: Проверка, является ли сторона \texttt{b} самой длинной.
  \item \textbf{S3 $\rightarrow$ S33}: Сторона \texttt{c} уже является самой длинной.
  \item \textbf{S31 $\rightarrow$ S41}: Сортировка сторон завершена (меняем \texttt{a} и \texttt{c}).
  \item \textbf{S32 $\rightarrow$ S42}: Сортировка сторон завершена (меняем \texttt{b} и \texttt{c}).
  \item \textbf{S33 $\rightarrow$ S43}: Сортировка сторон завершена (сортировка не требуется).
  \item \textbf{S4x $\rightarrow$ S5x}: Условие прямоугольного треугольника выполнено.
  \item \textbf{S4x $\rightarrow$ S6x}: Условие остроугольного треугольника выполнено.
  \item \textbf{S4x $\rightarrow$ S7x}: Условие тупоугольного треугольника выполнено.
  \item \textbf{S5x $\rightarrow$ S8x1}: Печать результата <<Прямоугольный треугольник>>.
  \item \textbf{S6x $\rightarrow$ S8x2}: Печать результата <<Остроугольный треугольник>>.
  \item \textbf{S7x $\rightarrow$ S8x3}: Печать результата <<Тупоугольный треугольник>>.
  \item \textbf{S2 $\rightarrow$ S8}: Печать результата <<Это не треугольник>>.
\end{itemize}

\clearpage
\section{Граф переходов между состояниями модели}

Граф переходов между состояниями модели приведен на рисунке \ref{img:scheme.pdf}.
\imgs{scheme.pdf}{h!}{0.77}{Граф переходов между состояниями модели}{}

\clearpage
\section{Вывод}

В ходе выполнения лабораторной работы были получены навыки работы с языком Promela и инструментом SPIN, а также выполнены следующие задачи:
\begin{itemize}
  \item Разработан фрагмент кода на языке Promela, который демонстрирует проверку типа треугольника на основе длин его сторон.
  \item Описана модель, включающая функциональность процесса \texttt{check\_triangle\_type} и его взаимодействие с процессом \texttt{init}.
  \item Перечислено множество состояний процесса \texttt{check\_triangle\_type} с текстовым пояснением причин переходов между состояниями.
  \item Построен граф переходов между состояниями модели, иллюстрирующий возможные пути выполнения процесса.
\end{itemize}

\end{document}
