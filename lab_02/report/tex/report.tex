\documentclass{bmstu}

\begin{document}

\makereporttitle
{Информатика и системы управления (ИУ)}
{Программное обеспечение ЭВМ и информационные технологии (ИУ7)}
{\textbf{2}}
{Математические основы верификации ПО}
{Моделирование гонки процессов}
{}
{ИУ7-42М}
{К.Э. Ковалец}
{О.В. Кузнецова}


\setcounter{page}{2}
% \renewcommand{\contentsname}{Содержание} 
% \tableofcontents


\chapter{Выполнение лабораторной работы}

\section{Задание}

Необходимо описать взаимодействие двух процессов, работающих с одними данными. Затем место возникновения гонки необходимо дополнить мьютексами.

Отчёт должен содержать:
\begin{itemize}
  \item описание модели взаимодействия процессов;
  \item демонстрация логов SPIN, в которых видна гонка;
  \item описание модели с мьютексом;
  \item результат корректного взаимодействия процессов --- логи SPIN;
  \item выводы по работе.
\end{itemize}

\section{Описание модели взаимодействия процессов}

Модель взаимодействия процессов описывает два параллельных процесса, которые работают с общей переменной \texttt{shared\_data}. Один процесс (\texttt{increase\_counter}) увеличивает значение этой переменной на \texttt{INCREMENT}, другой (\texttt{reduce\_counter}) уменьшает. 

В цикле, который выполняется \texttt{REPETIONS\_NUMBER} раз, оба процесса считывают текущее значение \texttt{shared\_data} в переменную \texttt{temp}, изменяют ее на \texttt{INCREMENT} и записывают обратно в \texttt{shared\_data}.

В данной модели возникает гонка, так как оба процесса одновременно обращаются к общей переменной \texttt{shared\_data} без синхронизации, что может привести к некорректным результатам.

\clearpage
Код программы c гонкой процессов на языке Promela приведен в листинге \ref{lst:race}.

\mylisting[c]{race.pml}{}{Пример программы на языке Promela c гонкой процессов}{race}{}

\section{Демонстрация логов SPIN, в которых видна гонка}

Демонстрация логов SPIN, в которых видна гонка, приведена в листинге \ref{lst:raceRes}. Можно заметить, что оба процесса одновременно считывают одно и то же начальное значение переменной и затем независимо изменяют его. В итоге каждый процесс записывает своё значение обратно в переменную \texttt{shared\_data}, что приводит к некорректному результату.

\mylisting[text]{race.txt}{}{Демонстрация логов SPIN, в которых видна гонка}{raceRes}{}

\section{Описание модели с мьютексом}

Модель с мьютексом добавляет синхронизацию между процессами, чтобы избежать гонки процессов. В данной модели используется мьютекс для обеспечения взаимного исключения при доступе к общей переменной \texttt{shared\_data}.

В цикле, который выполняется \texttt{REPETIONS\_NUMBER} раз, оба процесса по очереди блокирует доступ к критической секции, считывают текущее значение \texttt{shared\_data} в переменную \texttt{temp}, изменяют \texttt{temp} на \texttt{INCREMENT}, записывают обратно в \texttt{shared\_data} и разблокируют критическую секцию.

Использование мьютекса гарантирует, что только один процесс в каждый момент времени может изменять значение \texttt{shared\_data}, что предотвращает возникновение гонки процессов и обеспечивает корректное их взаимодействие.

Код программы c мьютексом на языке Promela приведен в листинге \ref{lst:mutex}.

\mylisting[c]{mutex.pml}{}{Пример программы на языке Promela c мьютексом}{mutex}{}

\section{Результат корректного взаимодействия процессов}

Результат корректного взаимодействия процессов приведен в листинге \ref{lst:mutexRes}. Можно заметить, что оба процесса последовательно получают доступ к критической секции (поочередно считывают, изменяют и записывают значение переменной \texttt{shared\_data}), что гарантирует корректное взаимодействие процессов.

\mylisting[text]{mutex.txt}{}{Результат корректного взаимодействия процессов --- логи SPIN}{mutexRes}{}

\section{Вывод}

В ходе выполнения лабораторной работы была разработана модель взаимодействия двух процессов, работающих с общей переменной. Была продемонстрирована гонка процессов, возникающая при одновременном доступе к критической секции без синхронизации. Для предотвращения гонки процессов была разработана модель с использованием мьютекса, который обеспечивает взаимное исключение при доступе к общей переменной. Результаты работы программ были продемонстрированы с помощью логов SPIN.  

\end{document}
