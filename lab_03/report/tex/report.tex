\documentclass{bmstu}

\begin{document}

\makereporttitle
{Информатика и системы управления (ИУ)}
{Программное обеспечение ЭВМ и информационные технологии (ИУ7)}
{\textbf{3}}
{Математические основы верификации ПО}
{Моделирование сетевого протокола}
{}
{ИУ7-42М}
{К.Э. Ковалец}
{О.В. Кузнецова}


\setcounter{page}{2}
% \renewcommand{\contentsname}{Содержание} 
% \tableofcontents


\chapter{Выполнение лабораторной работы}

\section{Задание}

Выбирается любой сетевой протокол и описывается упрощенная модель этого протокола.
Необязательно полностью все поля, например, IP-пакетов.

В отчёте содержится:
\begin{itemize}
  \item описание протокола и принятые допущения;
  \item описываемые uml-sequence при работе;
  \item модель протокола;
  \item логи SPIN, демонстрирующие отправку/получение данных: пакетов, HTML-документов и т.д.;
  \item выводы по работе.
\end{itemize}

\section{Описание протокола и принятые допущения}

В данной работе разработана и реализована упрощённая модель транспортного протокола TCP (Transmission Control Protocol). TCP является протоколом транспортного уровня, который обеспечивает надёжную и упорядоченную передачу данных между двумя узлами в сети. Для упрощения модели были приняты следующие допущения:
\begin{itemize}
  \item исключены сложные механизмы управления потоком, перегрузкой и повторной передачей данных;
  \item передача данных осуществляется по принципу Stop-and-Wait, то есть по одному пакету за раз с обязательным подтверждением;
  \item используются только основные типы сообщений: \texttt{AUTH} (авторизация), \texttt{OK} (подтверждение), \texttt{REQ} (запрос данных), \texttt{DATA} (данные), \texttt{ACK} (подтверждение получения данных) и \texttt{FIN} (завершение соединения);
  \item канал связи моделируется с ограниченным буфером, способным хранить не более трёх сообщений одновременно;
  \item отсутствует обработка ошибок, таких как потеря или повреждение пакетов.
\end{itemize}

Модель демонстрирует основные этапы взаимодействия между клиентом и сервером, включая установление соединения, передачу данных и завершение соединения. Такое упрощение позволяет сосредоточиться на базовых принципах работы протокола TCP, не углубляясь в сложные аспекты его реализации.

\section{Описываемые UML-схемы при работе}

На рисунке \ref{img:diagram.pdf} показана диаграмма последовательности, иллюстрирующая процесс взаимодействия между получателем и отправителем в рамках упрощённой модели протокола TCP. Диаграмма отображает основные этапы работы протокола, включая установление соединения, передачу данных и завершение соединения.

\imgs{diagram.pdf}{h!}{1.4}{Диаграмма последовательности упрощенного протокола TCP}{}

\section{Модель протокола}

В данной работе была разработана и реализована модель упрощённого транспортного протокола TCP с использованием языка Promela. Код модели приведён в листинге \ref{lst:protocolModel}.

Модель позволяет наглядно продемонстрировать основные принципы работы протокола TCP, включая обмен сообщениями и управление последовательностью взаимодействий между процессами.

Система включает два процесса: \texttt{Sender} (отправитель) и \texttt{Receiver} (получатель), которые взаимодействуют посредством канала \texttt{ch}. Для обмена данными используются заранее определённые типы сообщений: \texttt{AUTH}, \texttt{OK}, \texttt{REQ}, \texttt{DATA}, \texttt{ACK} и \texttt{FIN}.

Основные этапы взаимодействия между отправителем и получателем, которые демонстрирует модель, включают:

\begin{itemize}
  \item Установление соединения, где получатель инициирует авторизацию (\texttt{AUTH}), а отправитель подтверждает её (\texttt{OK}).
  \item Запрос данных, где получатель отправляет запрос (\texttt{REQ}), а отправитель начинает передачу данных (\texttt{DATA}).
  \item Передачу данных, осуществляемую по принципу Stop-and-Wait: отправитель отправляет данные (\texttt{DATA}), а получатель подтверждает их получение (\texttt{ACK}).
  \item Завершение соединения, при котором получатель инициирует завершение (\texttt{FIN}), а отправитель подтверждает его (\texttt{OK}).
\end{itemize}

\mylisting[c]{main.pml}{}{Код с реализацией программы для демонстрации работы упрщенного протокола TCP}{protocolModel}{}

\section{Логи SPIN, демонстрирующие отправку/получение данных}

Логи SPIN демонстрируют последовательность взаимодействий между процессами \texttt{Sender} и \texttt{Receiver}. В логах видно, что процессы корректно обмениваются сообщениями в соответствии с описанным протоколом.

Логи SPIN приведены в листинге \ref{lst:spinLogs}.

\mylisting[text]{result.txt}{}{Ллоги SPIN, демонстрирующие отправку/получение данных}{spinLogs}{}

\section{Вывод}

В ходе выполнения лабораторной работы была разработана и реализована упрощённая модель транспортного протокола TCP с использованием языка Promela. Модель продемонстрировала основные принципы работы протокола, включая установление соединения, передачу данных и завершение соединения. Логи SPIN демонстрируют корректное взаимодействие процессов \texttt{Sender} и \texttt{Receiver}, а также отсутствие взаимных блокировок.
\end{document}
